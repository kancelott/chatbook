% !TEX TS-program = xelatex
% !TEX encoding = UTF-8

% =================================
% whatsbook.tex
% Created by Pelle Beckman, 2016
% Modified by Kancelot To, 2019
%
% See medium.com/@pbeck and the post about
% 'Books from WhatsApp' for more info. 
% =================================

%
% Document setup
%
\documentclass[9pt, openany]{memoir} % Add/remove 'showtrims' to show trims
\let\providelength\undefined
\let\providecounter\undefined

%\usepackage{showframe} % Great for "debugging" layouts

%
% Title & Author
%
\title{Title}
\author{Author}

%
% Colours
%
\usepackage[dvipsnames]{xcolor}
\definecolor{color1}{RGB}{33, 33, 33}
\definecolor{color2}{RGB}{90, 181,191}
\definecolor{color3}{RGB}{135, 191, 120}
\definecolor{lightgray}{gray}{0.1}

%
% Type
%
\usepackage{blindtext}
\usepackage[UKenglish]{babel}
\usepackage[UKenglish]{isodate}
\usepackage{xltxtra} % Extra customizations for XeLaTeX
\usepackage{xunicode} % Unicode support for LaTeX character names (accents, European chars, etc)
\usepackage[hyphens]{url}
\usepackage[pdfpagelayout=TwoPageRight]{hyperref}
\hypersetup{%
	colorlinks=true,
	linkcolor=RoyalBlue,
	urlcolor=RoyalBlue,
	allbordercolors={0 0 0}
}
\usepackage{microtype} % Improves character and word spacing
\usepackage{xeCJK}
\usepackage{titlesec}

\cleanlookdateon

%
% Stock & page size
%
\setstocksize{8.25in}{5.125in}
\settrimmedsize{8in}{5in}{*}
\settrims{0.125in}{0.125in}
\setlrmarginsandblock{0.73in}{0.54in}{*}
\setulmarginsandblock{0.45in}{0.75in}{*}
%\setstocksize{9.25in}{6.125in}
%\settrimmedsize{9in}{6in}{*}
%\settrims{0.125in}{0.125in}
%\setlrmarginsandblock{20mm}{18mm}{*}
%\setulmarginsandblock{16mm}{28mm}{*}
\checkandfixthelayout

%
% Font
%
\setmainfont[Ligatures=TeX,
             BoldFont = {Ideal Sans Semibold},
             BoldItalicFont = {Ideal Sans Semibold Italic}]{Ideal Sans Book}
\setmonofont[Scale=MatchLowercase]{Consolas}
\setCJKmainfont{Noto Sans CJK TC Regular}

% Remove hypenation, see
\usepackage[none]{hyphenat}

%
% Chapters
%
\usepackage{pagecolor}
\usepackage[pages=some]{background}

\backgroundsetup{
    scale=1,
    angle=0,
    opacity=1,
    contents={\includegraphics[width=\stockwidth,height=\stockheight]{a45b24635eb16823062050dd2f807369}}
}
\makechapterstyle{WABChapter}{%
    \renewcommand*{\chapterheadstart}{\vspace*{140pt}}
    \renewcommand{\chaptitlefont}{\huge \centering \fontspec[Letters=Uppercase, Scale=5]{Tungsten Medium}}
    \renewcommand{\printchaptertitle}[1]{%
        \chaptitlefont \color{black}\MakeUppercase{{##1}}
        \vfill
    }
    \renewcommand{\printchaptername}{ }
    \renewcommand{\chapternamenum}{ }
    \renewcommand{\printchapternum}{ }
}
\chapterstyle{WABChapter}

%
% Sections
%
\usepackage{titlesec}
\def\dotfill#1{\cleaders\hbox to #1{.}\hfill}
\newcommand\dotline[2][.5em]{\leavevmode\hbox to #2{\dotfill{#1}\hfil}}

\titleformat{\section}[display]
{\filcenter\normalfont\fontspec[Numbers={Lining}, Scale=0.86]{Ideal Sans Semibold}}
%{\thesection.}{0}{}[\dotline{0.3\linewidth}]
{\thesection.}{0}{}[\rule{0.25\linewidth}{.35ex}]

\titlespacing{\section}{0pt}{6pt}{0pt}

%
% Footers & Headings
%
\usepackage{nameref}
\makeatletter
\newcommand*{\currentname}{\@currentlabelname}
\makeatother

\makeevenfoot{headings}{\scriptsize\fontspec[Numbers={Lining}]{Ideal Sans Book}\thepage}{}{}
\makeoddfoot{headings}{}{}{\scriptsize\fontspec[Numbers={Lining}]{Ideal Sans Book}\thepage}
\makeevenhead{headings}{}{}{}
\makeoddhead{headings}{}{}{}

\copypagestyle{chapter}{plain} % make chapter a page style of its own
\makeevenfoot{chapter}{}{}{}
\makeoddfoot{chapter}{}{}{}
\makeevenhead{chapter}{}{}{}
\makeoddhead{chapter}{}{}{}

%
% User styling
%
\newcommand{\kancelot}[0]{\textsc{kt}}
\newcommand{\cecilia}[0]{\textsc{cc}}

%
% Content
%
\usepackage{marginnote}
\renewcommand*{\marginfont}{\footnotesize\color{gray}}

\usepackage{longtable}
\newcolumntype{R}{r>{\raggedright\arraybackslash}}

\usepackage{epigraph}
\setlength{\epigraphrule}{0pt}
\setlength{\epigraphwidth}{0.3\textwidth}

\usepackage{pgfornament}


%
% Images
%
\graphicspath{ {./images/} }
\usepackage{float}
\usepackage[export]{adjustbox} % For frames
\usepackage{caption}

%
% Emojis
%
\usepackage{ifxetex}
\usepackage{ifluatex}
\ifxetex
	\usepackage[ios,font=seguiemj.ttf]{emoji}
	\usepackage{fontspec}
\else
	\ifluatex
		\usepackage[ios,font=Symbola_hint.ttf]{emoji}
		\usepackage{fontspec}
	\else
		\usepackage[T1]{fontenc}
		\usepackage[utf8]{inputenc}
		\usepackage[ios]{emoji}
	\fi
\fi

% PDF/X-3 stuff, necessary for Blurb IF USING xelatex
\special{pdf:docinfo <<
    /Title (Title)                                          % set your title here
    /Author (Authors)                                       % set author name
    /Subject (Everyday conversation)                        % set subject
    /Keywords (everyday conversation, chat logs, memoir)    % set keywords
    /Trapped (False)
    /GTS_PDFXVersion (PDF/X-3:2002)
    % must have a trim box, but I think Blurb ignores the values
    % /TrimBox [0.00000 9.00000 684.36000 585.00000] >>
}
\special{pdf:put @catalog <<
    /OutputIntents [ <<
    /Info (none)
    /Type /OutputIntent
    /S /GTS_PDFX
    /OutputConditionIdentifier (Blurb.com)
    /RegistryName (http://www.color.org/)
    >> ] >>
}

%
% Page breaks
%
%\raggedbottom

%
% Document
%
\usepackage{newclude}

\begin{document}

\OnehalfSpacing

\frontmatter
%
% p1 - half-title
%
\thispagestyle{empty}
\vspace*{2cm}
\begin{center}
    {\LARGE TITLE}
\end{center}
\clearpage


%
% p2 - blank (frontispiece)
%
\thispagestyle{empty}
\vspace*{0.05cm}

\begin{figure}[H]
	\begin{minipage}[b]{0.36\textwidth}
		\raggedleft
    	Yours truly, the co-author, who did all this hard work ordering this book.
	\end{minipage}\hfill
	\begin{minipage}[b]{0.61\textwidth}
	\includegraphics[width=\textwidth,keepaspectratio]{frontispiece-kt-20190927_135511.jpg}
	\end{minipage}
\end{figure}

\vspace*{0.5cm}

\begin{figure}[H]
	\begin{minipage}[b]{0.36\textwidth}
	\raggedleft
	And the other co-author, who did nothing.
	\end{minipage}\hfill
	\begin{minipage}[b]{0.61\textwidth}
	\includegraphics[width=\textwidth,keepaspectratio]{frontispiece-cc-20190413_161545.jpg}
	\end{minipage}
\end{figure}
\clearpage


%
% p3 - title
%
\thispagestyle{empty}
\vspace*{2cm}
\begin{center}
    {\Huge Title} \\

    \vspace{6ex}

    {\large\itshape A glimpse into all our conversations for your stalking pleasure.} \\

    \vspace{9cm}

    {\LARGE Kancelot To} \\
    \vspace{1.7ex}
    {\Large\fontspec{Garamond Premier Pro}\itshape \&} \\
    \vspace{2.5ex}
    {\LARGE Person A} \\
\end{center}
\clearpage


%
% p4 - colophon (copyright, colophon)
%
\thispagestyle{empty}
\vspace*{13cm}
\begin{center}
    \small

    Compiled with \XeLaTeX{} after much data wrangling on Kancelot's PC.

    Typeset in \emph{Ideal Sans} with chapter titles set in \emph{Tungsten}.

    \vspace{1.2ex}

    Typesetting, cover design, emoji and \emph{LINE} sticker extractions by Kancelot To.

    \vspace{2ex}
    \pgfornament[width=0.75cm,symmetry=v]{3}\pgfornament[width=0.75cm]{3}
    \vspace{3ex}

    Limited edition, this copy being 1 / 1.

    \vspace{1.2ex}

    \today
\end{center}
\clearpage


%
% p5 - dedication
%
\thispagestyle{empty}
\vspace*{2cm}
\begin{center}
	{\Large\itshape Dedicated to Santa Claus}
	
	\vspace{4ex}
	
	{\itshape for Christmas -- December 2019}
\end{center}
\clearpage


%
% p6 - blank
%
\begingroup
\pagestyle{empty}
\cleardoublepage
\endgroup


%
% p7 - timeline
%
\thispagestyle{empty}
\vspace*{2cm}
{\Huge Timeline} \\

\vspace{3cm}

\begin{longtable}{Rp{0.7\linewidth}}

 1 March 2018 & This happened \\[1ex]

 1 March 2018 & And then this happened \\[1ex]

 1 March 2018 & And this \\[1ex]

 1 March 2018 & Then that \\[1ex]

 1 March 2018 & The end \\[1ex]

\end{longtable}

\clearpage

%
% p8 - blank
%
\begingroup
\pagestyle{empty}
\cleardoublepage
\endgroup


\mainmatter
\include{whatsbook-folio}

\end{document}
